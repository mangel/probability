\documentclass{article}

\usepackage{amsmath}
\usepackage{amsfonts}
\usepackage{amssymb}
\usepackage[spanish, mexico]{babel}
\usepackage[margin=0.5in]{geometry}
\usepackage[utf8]{inputenc}

\begin{document}
\title{Taller de Distribución Normal}
\author{Maykoll J. Martínez, Miguel A. Gómez}
\maketitle

\paragraph{} Suponga que el ingreso en \textbf{en miles de pesos}, de los habitantes de tres ciudades, se encuentran normalmente distribuidos conforme a los siguientes parámetros:

\begin{center}
	\begin{tabular}{ | l | r | r | r | }
		\hline
		\textbf{Ciudad} & \textbf{$\mu$} & \textbf{$\sigma$} & \textbf{Personas} \\
		\hline
		Ciudad A & 1200 & 160 & 85.000 \\
		\hline
		Ciudad B & 1400 & 200 & 120.000 \\
		\hline
		Ciudad C & 1600 & 300 & 160.000 \\
		\hline
	\end{tabular}
\end{center}

El gobierno central, decide aplicar una serie de subsidios e impuestos en cada una de las ciudades conforme a los siguientes criterios (datos en miles de pesos):

\begin{center}
	\begin{tabular}{ | l | c | c | }
		\hline
		\space & \textbf{Subsidio} & \textbf{Impuesto} \\
		\hline
		\textbf{Ciudad} & Ingresos menores a & Ingresos mayores a \\
		\hline
		Ciudad A & 880 & 1600\\
		\hline
		Ciudad B & 900 & 1920\\
		\hline
		Ciudad C & 940 & 2200\\
		\hline
	\end{tabular}
\end{center}

De igual manera, el monto de subsidios e impuestos es diferente en cada ciudad, conforme a la siguiente tabla:

\begin{center}
	\begin{tabular}{ | l | c | c | }
		\hline
		\multicolumn{3}{ | c | }{\textbf{Valor(miles de pesos)}}\\
		\hline
		\textbf{Ciudad} & \textbf{Subsidio} & \textbf{Impuesto} \\
		\hline
		Ciudad A & 15 & 25 \\
		\hline
		Ciudad B & 18 & 23 \\
		\hline
		Ciudad C & 16 & 12 \\
		\hline
	\end{tabular}
\end{center}

\paragraph{Preguntas.}

\begin{enumerate}
	\item Cuántas personas serán beneficiadas del subisidio en las tres ciudades?
	\item Cuántas personas serán afectadas por el impuesto en las tres ciudades?
	\item Si el gobierno decide financiar los subsidios otorgado, con los impuestos pagados, en cual de las tres ciudades los recursos recogidos por la administración alcanzan para sufragar el costo de los subsidios?
	\item Al considerar las tres ciudades, en total el gobierno resulta con superávit o con déficit? Cuánto? Expresar su respuesta en miles de pesos.
	\item Suponga que el gobierno central desea realizar una estimación del total de dinero que recibiría en impuestos si se decide por aplicar impuestos al 12\% de las personas con ingresos más altos en las tres ciudades y una estimación del total de los subsidios que debería desembolsar, si decide otorgar subsidios al 15\% de las personas con menos ingresos en las tres ciudades. En este escenario, al considerar las tres ciudades, el gobierno tiene déficit o superavit? Justifique su respuesta.
	\item Con relación al punto 5, en cada una de las tres ciudades, cual sería el ingreso máximo de una persona para ser beneficiaria del subsidio y cual sería el ingreso mínimo para pagar el impuesto?
\end{enumerate}

\paragraph{Solución 1} En general, para las tres ciudades debemos obtener el valor de z para las poblaciones, posterior a ello obtener el valor de probabilidady multiplicarlo por el número de habitantes de cada ciudad.

\paragraph{Para A.} en la tabla 2, encontramos el valor límite al que un habitante de la ciudad A puede aplicar al subsidio.
$$Z = \frac{880 - 1200}{160} = -2 $$

Por ende $P(\text{salario} \leq 880) \approx 0.0228 = 2.28\%$ y la cantidad de  población sera el producto de esta probabilidad por el número de habitantes de la ciudad A, es decir: 

$$85000*2.28\% = 1938$$

1938 habitantes de la ciudad A serían beneficiados con el subsidio.

\paragraph{} Análogamente se realiza el mismo con las demás ciudades y obtenemos la siguiente tabla:


\begin{center}
	\begin{tabular}{ | c | c | c | c | c | c | c | c | }
		\hline
		\textbf{Ciudad} & \textbf{Restricción Ingresos} & \textbf{$\mu$} & \textbf{$\sigma$} & \textbf{Z} & \textbf{$P(X)$} & \textbf{Población} & \textbf{Beneficiarios}\\
		\hline
		Ciudad A & 880 & 1200 & 160 & -2 & 0.228 & 85000 & 1938 \\
		\hline
		Ciudad B & 900 & 1400 & 200 & -2.5 & 0.0062 & 120000 & 744 \\
		\hline
		Ciudad C & 940 & 1600 & 300 & -2.2 & 0.0139  & 160000 & 2224 \\
		\hline
	\end{tabular}
\end{center}

\paragraph{Solución 2} En general, seguimos el mismo procedimiento que en el punto anterior.

\paragraph{Para A.} en la tabla 2, encontramos el valor límite al que un habitante de la ciudad A debe pagar impuestos.
$$Z = \frac{1600 - 1200}{160} =  2.5$$

Sin embargo, para este caso necesitamos los valores mayores a nuestra restricción, por ende tomamos el valor inverso de Z, es decir $Z = -2.5$. Por ende $P(\text{salario} \geq 1600) \approx 0.0062 = 0.62\%$ y la cantidad de  población será el producto de esta probabilidad por el número de habitantes de la ciudad A, es decir: 

$$85000*0.62\% = 527$$

527 habitantes de la ciudad A tendrían que pagar impuestos.

\paragraph{} Análogamente se realiza el mismo con las demás ciudades y obtenemos la siguiente tabla:


\begin{center}
	\begin{tabular}{ | c | c | c | c | c | c | c | c | }
		\hline
		\textbf{Ciudad} & \textbf{Restricción Ingresos} & \textbf{$\mu$} & \textbf{$\sigma$} & \textbf{Z} & \textbf{$P(X)$} & \textbf{Población} & \textbf{Contribuyentes}\\
		\hline
		Ciudad A & 1600 & 1200 & 160 & -2.5 & 0.0062 & 85000 & 527 \\
		\hline
		Ciudad B & 1920 & 1400 & 200 & -2.6 & 0.0047 & 120000 & 564 \\
		\hline
		Ciudad C & 2200 & 1600 & 300 & -2 & 0.0228 & 160000 & 3648 \\
		\hline
	\end{tabular}
\end{center}

\paragraph{Solución 3} Debemos calcular los montos totales de impuestos y subisidos en cada ciudad, posterior a ello evaluar si la cantidad de impuestos recolectados es mayor a la de subsidios otorgados, de ser así, con los impuestos se podrían financiar los subsidios en dicha ciudad.

\paragraph{} Definimos las variables:

\begin{itemize}
	\item $P_{Ij}$: La cantidad de personas que deben pagar impuestos en la ciudad $j$.
	\item $P_{Sj}$: La cantidad de personas que serán beneficiarias del subsidio del gobierno en la ciudad $j$.
	\item $T_{Ij}$: El total de dinero en impuestos que serán pagados en la ciudad $j$.
	\item $T_{Sj}$: El total de dinero en subsidios que serán entregados en la ciudad $j$.
	\item $I_j$: El valor que se paga en impuestos en la ciudad $j$.
	\item $S_j$: El valor del subsidio que se otorga en la ciudad $j$.
\end{itemize}

Generalizando para cualquier ciudad tenemos dos fórmulas:

$$T_{Sj} = P_{Sj}S_j$$
$$T_{Ij} = P_{Ij}I_j$$

Aplicando la fórmula, obtenemos:

\begin{center}
	\begin{tabular}{ | c | c | c | c | c | c | c | c | }
		\hline
		\textbf{Ciudad} & \textbf{$I_j$} & \textbf{$S_j$} & \textbf{$P_{Ij}$} & \textbf{$P_{Sj}$} & \textbf{$T_{Sj}$} & \textbf{$T_{Ij}$} & \textbf{Alcanza?}\\
		\hline
		Ciudad A & 25 & 15 & 527 & 1938 & 29070 & 13175 & NO \\
		\hline
		Ciudad B & 23 & 18 & 564 & 744 & 13392 & 12972 & NO \\
		\hline
		Ciudad C & 12 & 16 & 3648 & 2224 & 35584 & 43776 & SI \\
		\hline
	\end{tabular}
\end{center}

Por lo tanto, únicamente en la ciudad C se puede sufragar el subsidio con lo recogido en impuestos.

\paragraph{Solución 4} En este caso debemos realizar dos sumas, la suma del total de subsidios y la suma del total recolectado en impuestos, respectivamente:

$$T_S = 78046$$
$$T_I = 69923$$

Al realizar la diferencia entre los impuestos y lo que seentregaría en subsidios encontramos que:

$$R = T_I - T_S = 69923 - 78046 = -8123$$

Por lo tanto, si se decide sufragar de los impuestos los subisidios, el gobierno se encontrará en déficit por un monto de $-8123$ miles de pesos.

\paragraph{Solución 5.} En este caso, debido a que en escencia realizamos productos sucesivos, y dado que se deben preservar las constantes de proporción de $15\%$ y $12\%$ para la población que recibiría subsidios y para la población que debería pagar impuestos respectivamente. Únicamente es necesario multiplicar las dos constantes con los totales de subsidios e impuestos encontrados en el punto anterior. Finalmente incorporar estos valores en la ecuación que nos arroja el resultado:

\begin{align*}
	R' = T_{I} * 12\%  - T_{S} * 15\% &= 69923*12\% - 78046 * 15\%\\
	&\approx -3316.14 
\end{align*}

Por ende, aún aplicando estas constantes se obtiene que el gobierno continúa en déficit por aproximadamente $-3316.14$ miles de pesos.

\paragraph{Solución 6.} Para encontrar los valores de cada ciudad, debemos realizar el proceso inverso.

\paragraph{Para el caso de A} Por el ejercicio anterior, podemos establecer que el número de personas que podrían acceder al subsidio serían del 15\% de las que podían establecer en primera instancia, es decir:

$$1938*15\% = 290.7 $$

es decir casi 291 personas. Con este valor sustituimos en la ecuación con la que se estableció que para la ciudad A únicamente 1938 habitantes podrían acceder al subsidio, pero con el valor de probabilidad como incógnita:

$$P_{SA} = 85000 * P(X) = 290.7$$

Por ende,

$$P(X) = \frac{291}{85000} \approx 0.0034$$

Ahora buscamos su correspondiente valor Z, y encontramos que es $Z = -2.71$. Al reemplazar en el valor de la ecuación que determina Z obtenemos:

$$Z=-2.71 = \frac{x-1200}{160} \implies x = (-2.71*160) + 1200$$

$$ x = 766.4$$

Tal como se esperaba, al reducir el número de personas que pueden acceder al subsidio, encontramos que la barrera de acceso al beneficio es se incrementa decir, al pasar del 880 a 766.4, para la población de esta ciudad se deben poseer menos ingresos para acceder al beneficio. Otra manera de verlo sería que el 85\% de la población que inicialmente podía acceder al beneficio, ya no podría.

\paragraph{} Para los impuestos realizaremos el mismo proceso.

$$P_{IA} = 527 * 12\% = 63.24 \implies P(X) = \frac{63.24}{85000} \approx 0.0007 $$

Hallamos el valor de Z, y encontramos que es $Z = -3.02$, sin embargo, dada la restricción de pago de impuesto requiere que el salario sea mayor a cierto monto, debemos obtener el valor opuesto de Z, es decir $Z = 3.02$. Ahora reemplazando como en el ejercicio anterior tenemos:

$$Z = 3.02 = \frac{x-1200}{160} \implies x = (3.02*160) + 1200$$

$$x = 1683.2$$

Y como se esperaba, al disminuir el número de personas que pagarían impuestos, la barrera para el pago de impuestos también aumenta. Y siguiendo la misma interpretación del ejemplo del beneficio de subsidios, el 88\% de las personas que originalmente pagarían impuestos ya no deberían realizar el pago de impuestos.

\paragraph{} de manera análoga realizamos el mismo procedimiento con las demás ciudades y obtenemos la siguiente tabla

\begin{center}
	\begin{tabular}{ | l | c | c | }
		\hline
		\space & \textbf{Subsidio} & \textbf{Impuesto} \\
		\hline
		\textbf{Ciudad} & Ingresos menores a & Ingresos mayores a \\
		\hline
		Ciudad A & 766.4 & 1683.2 \\
		\hline
		Ciudad B & 798 & 2006 \\
		\hline
		Ciudad C & 736 & 2434 \\
		\hline
	\end{tabular}
\end{center}

\end{document}